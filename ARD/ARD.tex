\documentclass{article}
\usepackage[utf8]{inputenc}
\usepackage[margin=1in]{geometry}
\usepackage[open,openlevel=1]{bookmark}
\usepackage{fancyhdr}
\usepackage{gensymb}
\usepackage{hyperref}  % Allow embedded hyperlinks in text
\usepackage{multirow}  % Allow contents spanning multiple rows in table
\usepackage{adjustbox} % Ensure consistent width of tables in some sections
\usepackage{../Helix-OBC-Firmware/documentation/latex/doxygen}

% Remove "\+" framgents from Doxygen generated latex
\newcommand{\+}{}

% Make links not look like garbage
\hypersetup{
    colorlinks=true,
    linkcolor=black,
    filecolor=magenta,      
    urlcolor=cyan,
}

\newcommand{\projectName}{Avionics Reference Document}

\pagestyle{fancy}
\fancyhf{}
\lhead{\projectName}
\rhead{\includegraphics[width=1cm]{assets/LOGO-RocketProject-highres-trnsprnt-bckgd.png}}
\lfoot{\thepage}

\title{Avionics Reference Document}
\author{David Knight}
\date{\today}

\begin{document}

\maketitle
\addcontentsline{toc}{section}{Title Page}
\newpage
\tableofcontents
\addcontentsline{toc}{section}{Table of Contents}
\newpage
\listoftables
\newpage
\listoffigures
\newpage

\section{Introduction}
The Avionics Reference Document (ARD) is a high level document describing what the onboard avionics system will do. The onboard avionics system is refered to as The Helix System. It consists of multiple circuit boards called Extention Boards which can read data from sensors through the rocket or interact with actuators throughout the rocket. These Extention boards are all connected together in a ring topology which allows new Extention Boards to be added or removed from the system without impacting performace of the rest of the system. These Extention Boards all connect with a central board called the Onboard Computer (OBC) which collects data, makes decisions based on data, communicates with ground support systems, and records data.

\subsection{Purpose}
The purpose of this document is to outline for non avionics members a unified reference for how the onbaord computer will behave and why. Primary focus will be given to the different states that the rocket can be in (ie dry systems, leak check), what actions will be performed in each state, and what causes the transition between different states. This will be documented through extensive use of state machine diagrams.

Old version:\\
The purpose of this document is to act as a reference for how the OBC will behave. This includes:
\begin{enumerate}
    \item How the system will react to failures such as disconnected sensors and over-pressurization events. 
    \item The states the rocket can be in including dry systems, leak check, launch, or even a failure state.
    \item The ranges and accuracy of sensors in the system.
    \item The rate at which sensors will be measured at.
    \item System architecture and how data flows throughout the system.
\end{enumerate}

\subsection{Scope}
This document includes:
\begin{enumerate}
    \item How the system will react to failures such as disconnected sensors and over-pressurization events. 
    \item The states the rocket can be in including dry systems, leak check, launch, or even a failure state.
    \item The ranges and accuracy of sensors in the system.
    \item The rate at which sensors will be measured at.
    \item System architecture and how data flows throughout the system.
\end{enumerate}


Old version:\\
This document should be viewed from the perspective of a non-avionics member who wants to know what the onboard avionics system will do. This means outlining that steps between turning the system on to launch and recovery of the rocket. Failure modes and how to deal with errors are outlined in this document. The ARD also contains information regarding hardware that will be onboard the rocket, it's specifications, and links to the relevant datasheets. Where appropriate more detailed technical information is included as reference material to the engineers designing the system.

\subsection{Definitions and Acronyms}


\newpage

\pagebreak

% WARNING START OF SECTION AUTOGENERATED BY PYTHON SCRIPT
% THIS SECTION MAY BE AUTOMATICALLY CHANGED AT ANY TIME
% Autogenerated section name: Hardware Config
% File path of script: ARD/genHARDWARE.py
\section{Hardware}
\subsection{Pressure Transducers}
\begin{adjustbox}{width=0.75\linewidth}
\begin{tabular}{|p{0.3\linewidth}|p{0.5\linewidth}|}
\hline
Model Number & MLH05KPSB01G\\\hline
Serial Number & F8CEA38AA5\\\hline
Datasheet Link & \href{https://www.mouser.com/datasheet/2/187/honeywell-sensing-heavy-duty-pressure-mlh-series-d-1112521.pdf}{Link}\\\hline
Sensing Units & PSIG\\\hline
Pressure Port Type & 1/4-18 NPT (ANSI B1.20.1)\\\hline
Accuracy & $\pm0.25\%$\\\hline
Pressure Range & 0PSIG to 5000PSIG\\\hline
Data Frequency & 50Hz\\\hline
Output Voltage Range & 1.0 to 5.0 Volts\\\hline
Input Voltage Range & 8.0 to 30.0 Volts\\\hline
Temperature Range & -40\degree to +125\degree Celcius\\\hline
\end{tabular}
\end{adjustbox}
\newline
\vspace*{1em}
\newline
\begin{adjustbox}{width=0.75\linewidth}
\begin{tabular}{|p{0.3\linewidth}|p{0.5\linewidth}|}
\hline
Model Number & ASUHGP1K55A1AA1A20000\\\hline
Serial Number & E5C0ADEA35\\\hline
Datasheet Link & \href{https://www.kasensors.com/sites/default/files/downloads/ASUH.811_0.pdf}{Link}\\\hline
Sensing Units & PSIG\\\hline
Pressure Port Type & 3/8 Inch 24 UNF Dash 3 (SAE J514)\\\hline
Accuracy & $\pm0.25\%$\\\hline
Pressure Range & 0PSIG to 1500PSIG\\\hline
Data Frequency & 50Hz\\\hline
Output Voltage Range & 0.5 to 4.5 Volts\\\hline
Input Voltage Range & 8.0 to 16.0 Volts\\\hline
Temperature Range & -40\degree to +150\degree Celcius\\\hline
\end{tabular}
\end{adjustbox}
\newline
\vspace*{1em}
\newline
\subsection{Thermocouples}
\begin{adjustbox}{width=0.75\linewidth}
\begin{tabular}{|p{0.3\linewidth}|p{0.5\linewidth}|}
\hline
Model Number & 240-080\\\hline
Serial Number & BB510C3CE3\\\hline
Datasheet Link & \href{https://media.digikey.com/pdf/Data%20Sheets/Digilent%20PDFs/240-080_Web.pdf}{Link}\\\hline
Type & K\\\hline
Sensing Units & Celcius\\\hline
Data Frequency & 10Hz\\\hline
Temperature Range & -73\degree to +150\degree Celcius\\\hline
\end{tabular}
\end{adjustbox}
\newline
\vspace*{1em}
\newline
\subsection{RTDs}
\begin{adjustbox}{width=0.75\linewidth}
\begin{tabular}{|p{0.3\linewidth}|p{0.5\linewidth}|}
\hline
Model Number & 1PT100K2515\\\hline
Serial Number & 8105874731\\\hline
Datasheet Link & \href{https://assets.omega.com/spec/1PT100K-RTD-ELEMENTS.pdf}{Link}\\\hline
Type & PT100\\\hline
Sensing Units & Celcius\\\hline
Data Frequency & 10Hz\\\hline
Temperature Range & -200\degree to +150\degree Celcius\\\hline
\end{tabular}
\end{adjustbox}
\newline
\vspace*{1em}
\newline
\subsection{Hall Effect Sensors}
\begin{adjustbox}{width=0.75\linewidth}
\begin{tabular}{|p{0.3\linewidth}|p{0.5\linewidth}|}
\hline
Model Number & TCS40DPR\\\hline
Serial Number & 6D65BA9367\\\hline
Datasheet Link & \href{https://toshiba.semicon-storage.com/info/docget.jsp?did=30107&prodName=TCS40DPR}{Link}\\\hline
Sensing Units & mT\\\hline
Output Type & Push-Pull\\\hline
Trip & $\pm4.4$mT\\\hline
Release & $\pm0.9$mT\\\hline
Input Voltage Range & 8.0 to 16.0 Volts\\\hline
Data Frequency & 10Hz\\\hline
Temperature Range & -40\degree to +150\degree Celcius\\\hline
\end{tabular}
\end{adjustbox}
\newline
\vspace*{1em}
\newline
\newpage
% WARNING END OF SECTION AUTOGENERATED BY PYTHON SCRIPT
% THIS SECTION MAY BE AUTOMATICALLY CHANGED AT ANY TIME
% Autogenerated section name: Hardware Config
% File path of script: ARD/genHARDWARE.py

\section{States}
\subsection{STATE LEAK CHECK}
\subsubsection{Helium Pressure PT Data}



\begin{DoxyRetVals}{Return values}
{\em S\+T\+A\+T\+E\+\_\+\+L\+E\+A\+K\+\_\+\+C\+H\+E\+CK} & Continue in the leak check state. \\
\hline
{\em S\+T\+A\+T\+E\+\_\+\+I\+D\+LE} & Finished leak check so return to the idle state. \\
\hline
{\em S\+T\+A\+T\+E\+\_\+\+G\+R\+O\+U\+N\+D\+\_\+\+S\+A\+FE} & Return to the ground safe state because the helium tank pressure is overpressurized.\\
\hline
\end{DoxyRetVals}
When Helium Pressure PT Data is received the C\+A\+N\+ID will be printed to stdout and the data will be printed as a string to stdout. The current time and data with milliseconds is then printed to stdout. The received can\+\_\+frame is added to the event\+Timer so that the received frame will be receved again in 1 second. The system then continues on in the leak check state.


% WARNING START OF SECTION AUTOGENERATED BY PYTHON SCRIPT
% THIS SECTION MAY BE AUTOMATICALLY CHANGED AT ANY TIME
% Autogenerated section name: EEPROM Config
% File path of script: ARD/genEEPROM.py
\section{EEPROM Layouts}
\subsection{Layout Version IDs}
\begin{tabular}{ |l|l| }
\hline
VersionID & Version Name \\\hline
1 & Sensor Board Layout Rev 1 \\\hline
2 & Power Distro Board Layout Rev 1 \\\hline
\end{tabular}
\newpage

\subsection{Sensor Board Layout Rev 1}
{\tiny
\begin{adjustbox}{width=1\textwidth}
\begin{tabular}{ |p{0.05\linewidth}|p{0.15\linewidth}||p{0.05\linewidth}|p{0.15\linewidth}||p{0.05\linewidth}|p{0.15\linewidth}| }
\hline
\multicolumn{6}{|c|}{Sensor Board Layout Rev 1 Page \#0} \\\hline
Byte \# & Usage & Byte \# & Usage & Byte \# & Usage \\\hline
0 & \multirow{4}{0.9\linewidth}{\centering Layout Rev Number} & 48 & \multirow{4}{0.9\linewidth}{\centering PT0 Polyfit p2} & 96 & \multirow{4}{0.9\linewidth}{\centering PT1 Current CanID} \\ 
1 &  & 49 &  & 97 &  \\ 
2 &  & 50 &  & 98 &  \\ 
3 &  & 51 &  & 99 &  \\ \cline{1-2} \cline{3-4} \cline{5-6} 
4 & \multirow{4}{0.9\linewidth}{\centering EEPROM Layout Compile Time} & 52 & \multirow{4}{0.9\linewidth}{\centering PT0 Polyfit p3} & 100 & \multirow{4}{0.9\linewidth}{\centering PT1 Max Voltage} \\ 
5 &  & 53 &  & 101 &  \\ 
6 &  & 54 &  & 102 &  \\ 
7 &  & 55 &  & 103 &  \\ \cline{1-2} \cline{3-4} \cline{5-6} 
8 & \multirow{4}{0.9\linewidth}{\centering Board Status} & 56 & \multirow{4}{0.9\linewidth}{\centering PT0 Polyfit p4} & 104 & \multirow{4}{0.9\linewidth}{\centering PT1 Min Voltage} \\ 
9 &  & 57 &  & 105 &  \\ 
10 &  & 58 &  & 106 &  \\ 
11 &  & 59 &  & 107 &  \\ \cline{1-2} \cline{3-4} \cline{5-6} 
12 & \multirow{4}{0.9\linewidth}{\centering Board VIN Voltage CanID} & 60 & \multirow{4}{0.9\linewidth}{\centering PT0 Polyfit p5} & 108 & \multirow{4}{0.9\linewidth}{\centering PT1 Max Value} \\ 
13 &  & 61 &  & 109 &  \\ 
14 &  & 62 &  & 110 &  \\ 
15 &  & 63 &  & 111 &  \\ \cline{1-2} \cline{3-4} \cline{5-6} 
16 & \multirow{4}{0.9\linewidth}{\centering Board current CanID} & 64 & \multirow{4}{0.9\linewidth}{\centering PT0 Polyfit p6} & 112 & \multirow{4}{0.9\linewidth}{\centering PT1 Min Value} \\ 
17 &  & 65 &  & 113 &  \\ 
18 &  & 66 &  & 114 &  \\ 
19 &  & 67 &  & 115 &  \\ \cline{1-2} \cline{3-4} \cline{5-6} 
20 & \multirow{4}{0.9\linewidth}{\centering PT0 Data CanID} & 68 & \multirow{4}{0.9\linewidth}{\centering PT0 Polyfit p7} & 116 & \multirow{4}{0.9\linewidth}{\centering PT1 Polyfit p1} \\ 
21 &  & 69 &  & 117 &  \\ 
22 &  & 70 &  & 118 &  \\ 
23 &  & 71 &  & 119 &  \\ \cline{1-2} \cline{3-4} \cline{5-6} 
24 & \multirow{4}{0.9\linewidth}{\centering PT0 Current CanID} & 72 & \multirow{4}{0.9\linewidth}{\centering PT0 Biquad Filter b0} & 120 & \multirow{4}{0.9\linewidth}{\centering PT1 Polyfit p2} \\ 
25 &  & 73 &  & 121 &  \\ 
26 &  & 74 &  & 122 &  \\ 
27 &  & 75 &  & 123 &  \\ \cline{1-2} \cline{3-4} \cline{5-6} 
28 & \multirow{4}{0.9\linewidth}{\centering PT0 Max Voltage} & 76 & \multirow{4}{0.9\linewidth}{\centering PT0 Biquad Filter b1} & 124 & \multirow{4}{0.9\linewidth}{\centering PT1 Polyfit p3} \\ 
29 &  & 77 &  & 125 &  \\ 
30 &  & 78 &  & 126 &  \\ 
31 &  & 79 &  & 127 &  \\ \cline{1-2} \cline{3-4} \cline{5-6} 
32 & \multirow{4}{0.9\linewidth}{\centering PT0 Min Voltage} & 80 & \multirow{4}{0.9\linewidth}{\centering PT0 Biquad Filter b2} &  & \\ 
33 &  & 81 &  &  & \\ 
34 &  & 82 &  &  & \\ 
35 &  & 83 &  &  & \\ \cline{1-2} \cline{3-4} 
36 & \multirow{4}{0.9\linewidth}{\centering PT0 Max Value} & 84 & \multirow{4}{0.9\linewidth}{\centering PT0 Biquad Filter a1} &  & \\ 
37 &  & 85 &  &  & \\ 
38 &  & 86 &  &  & \\ 
39 &  & 87 &  &  & \\ \cline{1-2} \cline{3-4} 
40 & \multirow{4}{0.9\linewidth}{\centering PT0 Min Value} & 88 & \multirow{4}{0.9\linewidth}{\centering PT0 Biquad Filter a2} &  & \\ 
41 &  & 89 &  &  & \\ 
42 &  & 90 &  &  & \\ 
43 &  & 91 &  &  & \\ \cline{1-2} \cline{3-4} 
44 & \multirow{4}{0.9\linewidth}{\centering PT0 Polyfit p1} & 92 & \multirow{4}{0.9\linewidth}{\centering PT1 Data CanID} &  & \\ 
45 &  & 93 &  &  & \\ 
46 &  & 94 &  &  & \\ 
47 &  & 95 &  &  & \\ \cline{1-2} \cline{3-4} 
\hline
\end{tabular}\\
\end{adjustbox}
} % end tiny
\newpage

{\tiny
\begin{adjustbox}{width=1\textwidth}
\begin{tabular}{ |p{0.05\linewidth}|p{0.15\linewidth}||p{0.05\linewidth}|p{0.15\linewidth}||p{0.05\linewidth}|p{0.15\linewidth}| }
\hline
\multicolumn{6}{|c|}{Sensor Board Layout Rev 1 Page \#1} \\\hline
Byte \# & Usage & Byte \# & Usage & Byte \# & Usage \\\hline
128 & \multirow{4}{0.9\linewidth}{\centering PT1 Polyfit p4} & 176 & \multirow{4}{0.9\linewidth}{\centering PT1 Min Voltage} & 224 & \multirow{4}{0.9\linewidth}{\centering PT1 Biquad Filter b2} \\ 
129 &  & 177 &  & 225 &  \\ 
130 &  & 178 &  & 226 &  \\ 
131 &  & 179 &  & 227 &  \\ \cline{1-2} \cline{3-4} \cline{5-6} 
132 & \multirow{4}{0.9\linewidth}{\centering PT1 Polyfit p5} & 180 & \multirow{4}{0.9\linewidth}{\centering PT1 Max Value} & 228 & \multirow{4}{0.9\linewidth}{\centering PT1 Biquad Filter a1} \\ 
133 &  & 181 &  & 229 &  \\ 
134 &  & 182 &  & 230 &  \\ 
135 &  & 183 &  & 231 &  \\ \cline{1-2} \cline{3-4} \cline{5-6} 
136 & \multirow{4}{0.9\linewidth}{\centering PT1 Polyfit p6} & 184 & \multirow{4}{0.9\linewidth}{\centering PT1 Min Value} & 232 & \multirow{4}{0.9\linewidth}{\centering PT1 Biquad Filter a2} \\ 
137 &  & 185 &  & 233 &  \\ 
138 &  & 186 &  & 234 &  \\ 
139 &  & 187 &  & 235 &  \\ \cline{1-2} \cline{3-4} \cline{5-6} 
140 & \multirow{4}{0.9\linewidth}{\centering PT1 Polyfit p7} & 188 & \multirow{4}{0.9\linewidth}{\centering PT1 Polyfit p1} & 236 & \multirow{4}{0.9\linewidth}{\centering Hall Effect 0 Data CanID} \\ 
141 &  & 189 &  & 237 &  \\ 
142 &  & 190 &  & 238 &  \\ 
143 &  & 191 &  & 239 &  \\ \cline{1-2} \cline{3-4} \cline{5-6} 
144 & \multirow{4}{0.9\linewidth}{\centering PT1 Biquad Filter b0} & 192 & \multirow{4}{0.9\linewidth}{\centering PT1 Polyfit p2} & 240 & \multirow{4}{0.9\linewidth}{\centering Hall Effect 0 Current CanID} \\ 
145 &  & 193 &  & 241 &  \\ 
146 &  & 194 &  & 242 &  \\ 
147 &  & 195 &  & 243 &  \\ \cline{1-2} \cline{3-4} \cline{5-6} 
148 & \multirow{4}{0.9\linewidth}{\centering PT1 Biquad Filter b1} & 196 & \multirow{4}{0.9\linewidth}{\centering PT1 Polyfit p3} & 244 & \multirow{4}{0.9\linewidth}{\centering Hall Effect 1 Data CanID} \\ 
149 &  & 197 &  & 245 &  \\ 
150 &  & 198 &  & 246 &  \\ 
151 &  & 199 &  & 247 &  \\ \cline{1-2} \cline{3-4} \cline{5-6} 
152 & \multirow{4}{0.9\linewidth}{\centering PT1 Biquad Filter b2} & 200 & \multirow{4}{0.9\linewidth}{\centering PT1 Polyfit p4} & 248 & \multirow{4}{0.9\linewidth}{\centering Hall Effect 1 Current CanID} \\ 
153 &  & 201 &  & 249 &  \\ 
154 &  & 202 &  & 250 &  \\ 
155 &  & 203 &  & 251 &  \\ \cline{1-2} \cline{3-4} \cline{5-6} 
156 & \multirow{4}{0.9\linewidth}{\centering PT1 Biquad Filter a1} & 204 & \multirow{4}{0.9\linewidth}{\centering PT1 Polyfit p5} & 252 & \multirow{4}{0.9\linewidth}{\centering Hall Effect 2 Data CanID} \\ 
157 &  & 205 &  & 253 &  \\ 
158 &  & 206 &  & 254 &  \\ 
159 &  & 207 &  & 255 &  \\ \cline{1-2} \cline{3-4} \cline{5-6} 
160 & \multirow{4}{0.9\linewidth}{\centering PT1 Biquad Filter a2} & 208 & \multirow{4}{0.9\linewidth}{\centering PT1 Polyfit p6} &  & \\ 
161 &  & 209 &  &  & \\ 
162 &  & 210 &  &  & \\ 
163 &  & 211 &  &  & \\ \cline{1-2} \cline{3-4} 
164 & \multirow{4}{0.9\linewidth}{\centering PT2 Data CanID} & 212 & \multirow{4}{0.9\linewidth}{\centering PT1 Polyfit p7} &  & \\ 
165 &  & 213 &  &  & \\ 
166 &  & 214 &  &  & \\ 
167 &  & 215 &  &  & \\ \cline{1-2} \cline{3-4} 
168 & \multirow{4}{0.9\linewidth}{\centering PT2 Current CanID} & 216 & \multirow{4}{0.9\linewidth}{\centering PT1 Biquad Filter b0} &  & \\ 
169 &  & 217 &  &  & \\ 
170 &  & 218 &  &  & \\ 
171 &  & 219 &  &  & \\ \cline{1-2} \cline{3-4} 
172 & \multirow{4}{0.9\linewidth}{\centering PT1 Max Voltage} & 220 & \multirow{4}{0.9\linewidth}{\centering PT1 Biquad Filter b1} &  & \\ 
173 &  & 221 &  &  & \\ 
174 &  & 222 &  &  & \\ 
175 &  & 223 &  &  & \\ \cline{1-2} \cline{3-4} 
\hline
\end{tabular}\\
\end{adjustbox}
} % end tiny
\newpage

{\tiny
\begin{adjustbox}{width=1\textwidth}
\begin{tabular}{ |p{0.05\linewidth}|p{0.15\linewidth}||p{0.05\linewidth}|p{0.15\linewidth}||p{0.05\linewidth}|p{0.15\linewidth}| }
\hline
\multicolumn{6}{|c|}{Sensor Board Layout Rev 1 Page \#2} \\\hline
Byte \# & Usage & Byte \# & Usage & Byte \# & Usage \\\hline
256 & \multirow{4}{0.9\linewidth}{\centering Hall Effect 2 Current CanID} & 304 & \multirow{4}{0.9\linewidth}{\centering TC1 Biquad Filter a2} & 352 & \multirow{4}{0.9\linewidth}{\centering RTD1 Biquad Filter a2} \\ 
257 &  & 305 &  & 353 &  \\ 
258 &  & 306 &  & 354 &  \\ 
259 &  & 307 &  & 355 &  \\ \cline{1-2} \cline{3-4} \cline{5-6} 
260 & \multirow{4}{0.9\linewidth}{\centering TC0 Data CanID} & 308 & \multirow{4}{0.9\linewidth}{\centering RTD0 Data CanID} & 356 &  \\ 
261 &  & 309 &  & 357 &  \\ 
262 &  & 310 &  & 358 &  \\ 
263 &  & 311 &  & 359 &  \\ \cline{1-2} \cline{3-4} 
264 & \multirow{4}{0.9\linewidth}{\centering TC0 Biquad Filter b0} & 312 & \multirow{4}{0.9\linewidth}{\centering RTD0 Biquad Filter b0} & 360 &  \\ 
265 &  & 313 &  & 361 &  \\ 
266 &  & 314 &  & 362 &  \\ 
267 &  & 315 &  & 363 &  \\ \cline{1-2} \cline{3-4} 
268 & \multirow{4}{0.9\linewidth}{\centering TC0 Biquad Filter b1} & 316 & \multirow{4}{0.9\linewidth}{\centering RTD0 Biquad Filter b1} & 364 &  \\ 
269 &  & 317 &  & 365 &  \\ 
270 &  & 318 &  & 366 &  \\ 
271 &  & 319 &  & 367 &  \\ \cline{1-2} \cline{3-4} 
272 & \multirow{4}{0.9\linewidth}{\centering TC0 Biquad Filter b2} & 320 & \multirow{4}{0.9\linewidth}{\centering RTD0 Biquad Filter b2} & 368 &  \\ 
273 &  & 321 &  & 369 &  \\ 
274 &  & 322 &  & 370 &  \\ 
275 &  & 323 &  & 371 &  \\ \cline{1-2} \cline{3-4} 
276 & \multirow{4}{0.9\linewidth}{\centering TC0 Biquad Filter a1} & 324 & \multirow{4}{0.9\linewidth}{\centering RTD0 Biquad Filter a1} & 372 &  \\ 
277 &  & 325 &  & 373 &  \\ 
278 &  & 326 &  & 374 &  \\ 
279 &  & 327 &  & 375 &  \\ \cline{1-2} \cline{3-4} 
280 & \multirow{4}{0.9\linewidth}{\centering TC0 Biquad Filter a2} & 328 & \multirow{4}{0.9\linewidth}{\centering RTD0 Biquad Filter a2} & 376 &  \\ 
281 &  & 329 &  & 377 &  \\ 
282 &  & 330 &  & 378 &  \\ 
283 &  & 331 &  & 379 &  \\ \cline{1-2} \cline{3-4} 
284 & \multirow{4}{0.9\linewidth}{\centering TC1 Data CanID} & 332 & \multirow{4}{0.9\linewidth}{\centering RTD1 Data CanID} & 380 &  \\ 
285 &  & 333 &  & 381 &  \\ 
286 &  & 334 &  & 382 &  \\ 
287 &  & 335 &  & 383 &  \\ \cline{1-2} \cline{3-4} \cline{5-6} 
288 & \multirow{4}{0.9\linewidth}{\centering TC1 Biquad Filter b0} & 336 & \multirow{4}{0.9\linewidth}{\centering RTD1 Biquad Filter b0} &  & \\ 
289 &  & 337 &  &  & \\ 
290 &  & 338 &  &  & \\ 
291 &  & 339 &  &  & \\ \cline{1-2} \cline{3-4} 
292 & \multirow{4}{0.9\linewidth}{\centering TC1 Biquad Filter b1} & 340 & \multirow{4}{0.9\linewidth}{\centering RTD1 Biquad Filter b1} &  & \\ 
293 &  & 341 &  &  & \\ 
294 &  & 342 &  &  & \\ 
295 &  & 343 &  &  & \\ \cline{1-2} \cline{3-4} 
296 & \multirow{4}{0.9\linewidth}{\centering TC1 Biquad Filter b2} & 344 & \multirow{4}{0.9\linewidth}{\centering RTD1 Biquad Filter b2} &  & \\ 
297 &  & 345 &  &  & \\ 
298 &  & 346 &  &  & \\ 
299 &  & 347 &  &  & \\ \cline{1-2} \cline{3-4} 
300 & \multirow{4}{0.9\linewidth}{\centering TC1 Biquad Filter a1} & 348 & \multirow{4}{0.9\linewidth}{\centering RTD1 Biquad Filter a1} &  & \\ 
301 &  & 349 &  &  & \\ 
302 &  & 350 &  &  & \\ 
303 &  & 351 &  &  & \\ \cline{1-2} \cline{3-4} 
\hline
\end{tabular}\\
\end{adjustbox}
} % end tiny
\newpage

\subsection{Power Distro Board Layout Rev 1}
{\tiny
\begin{adjustbox}{width=1\textwidth}
\begin{tabular}{ |p{0.05\linewidth}|p{0.15\linewidth}||p{0.05\linewidth}|p{0.15\linewidth}||p{0.05\linewidth}|p{0.15\linewidth}| }
\hline
\multicolumn{6}{|c|}{Power Distro Board Layout Rev 1 Page \#0} \\\hline
Byte \# & Usage & Byte \# & Usage & Byte \# & Usage \\\hline
0 & \multirow{4}{0.9\linewidth}{\centering Board Status} & 48 &  & 96 &  \\ 
1 &  & 49 &  & 97 &  \\ 
2 &  & 50 &  & 98 &  \\ 
3 &  & 51 &  & 99 &  \\ \cline{1-2} 
4 & \multirow{4}{0.9\linewidth}{\centering Offboard Battery Voltage CANID} & 52 &  & 100 &  \\ 
5 &  & 53 &  & 101 &  \\ 
6 &  & 54 &  & 102 &  \\ 
7 &  & 55 &  & 103 &  \\ \cline{1-2} 
8 & \multirow{4}{0.9\linewidth}{\centering Offboard Battery Current CANID} & 56 &  & 104 &  \\ 
9 &  & 57 &  & 105 &  \\ 
10 &  & 58 &  & 106 &  \\ 
11 &  & 59 &  & 107 &  \\ \cline{1-2} 
12 & \multirow{4}{0.9\linewidth}{\centering Onboard Battery Voltage CANID} & 60 &  & 108 &  \\ 
13 &  & 61 &  & 109 &  \\ 
14 &  & 62 &  & 110 &  \\ 
15 &  & 63 &  & 111 &  \\ \cline{1-2} 
16 & \multirow{4}{0.9\linewidth}{\centering Onboard Battery Current CANID} & 64 &  & 112 &  \\ 
17 &  & 65 &  & 113 &  \\ 
18 &  & 66 &  & 114 &  \\ 
19 &  & 67 &  & 115 &  \\ \cline{1-2} 
20 & \multirow{4}{0.9\linewidth}{\centering Helix Loop CW Voltage CANID} & 68 &  & 116 &  \\ 
21 &  & 69 &  & 117 &  \\ 
22 &  & 70 &  & 118 &  \\ 
23 &  & 71 &  & 119 &  \\ \cline{1-2} 
24 & \multirow{4}{0.9\linewidth}{\centering Helix Loop CW Current CANID} & 72 &  & 120 &  \\ 
25 &  & 73 &  & 121 &  \\ 
26 &  & 74 &  & 122 &  \\ 
27 &  & 75 &  & 123 &  \\ \cline{1-2} 
28 & \multirow{4}{0.9\linewidth}{\centering Helix Loop CCW Voltage CANID} & 76 &  & 124 &  \\ 
29 &  & 77 &  & 125 &  \\ 
30 &  & 78 &  & 126 &  \\ 
31 &  & 79 &  & 127 &  \\ \cline{1-2} \cline{5-6} 
32 & \multirow{4}{0.9\linewidth}{\centering Helix Loop CCW Current CANID} & 80 &  &  & \\ 
33 &  & 81 &  &  & \\ 
34 &  & 82 &  &  & \\ 
35 &  & 83 &  &  & \\ \cline{1-2} 
36 &  & 84 &  &  & \\ 
37 &  & 85 &  &  & \\ 
38 &  & 86 &  &  & \\ 
39 &  & 87 &  &  & \\ 
40 &  & 88 &  &  & \\ 
41 &  & 89 &  &  & \\ 
42 &  & 90 &  &  & \\ 
43 &  & 91 &  &  & \\ 
44 &  & 92 &  &  & \\ 
45 &  & 93 &  &  & \\ 
46 &  & 94 &  &  & \\ 
47 &  & 95 &  &  & \\ 
\hline
\end{tabular}\\
\end{adjustbox}
} % end tiny
\newpage

% WARNING END OF SECTION AUTOGENERATED BY PYTHON SCRIPT
% THIS SECTION MAY BE AUTOMATICALLY CHANGED AT ANY TIME
% Autogenerated section name: EEPROM Config
% File path of script: ARD/genEEPROM.py

% WARNING START OF SECTION AUTOGENERATED BY PYTHON SCRIPT
% THIS SECTION MAY BE AUTOMATICALLY CHANGED AT ANY TIME
% Autogenerated section name: CAN Config
% File path of script: ARD/genCAN.py
\section{CAN IDs}
\subsection{CAN Bus Load Calculations}
The current CAN Bus config requires between 30812 bits and 36972 bits to be sent on the CAN bus every second.

\begin{flushleft}\begin{tabular}{|c|c|c|}\hline
	Frequency & Best Case & Worst Case \\\hline
	100KHz & 31.0\% & 37.0\%\\\hline
	250KHz & 12.0\% & 15.0\%\\\hline
	500KHz & 6.0\% & 7.000000000000001\%\\\hline
	1MHz & 3.0\% & 4.0\%\\\hline
\end{tabular}
\end{flushleft}\subsection{ID 0 - Clock Sync}
Frequency: 50Hz\\
\begin{tabular}{ |p{0.05\linewidth}|p{0.05\linewidth}|p{0.1\linewidth}|p{0.15\linewidth}|p{0.2\linewidth}|p{0.3\linewidth}| }\hline
Byte & Bit & Signed & Range & Units & Description\\\hline
0-3 &  & False & 0 to 4294967295 & Milliseconds & UTC time\\\hline
\end{tabular}
\subsection{ID 1 - Emergency Signal}
Frequency: 50Hz\\
\begin{tabular}{ |p{0.05\linewidth}|p{0.05\linewidth}|p{0.1\linewidth}|p{0.15\linewidth}|p{0.2\linewidth}|p{0.3\linewidth}| }\hline
Byte & Bit & Signed & Range & Units & Description\\\hline
0 &  & False &  &  & Status\\\hline
 & 0-1 & & & & System Status\\\hline
\end{tabular}
\subsection{ID 50 - Transition Cone Status}
Frequency: 10Hz\\
\begin{tabular}{ |p{0.05\linewidth}|p{0.05\linewidth}|p{0.1\linewidth}|p{0.15\linewidth}|p{0.2\linewidth}|p{0.3\linewidth}| }\hline
Byte & Bit & Signed & Range & Units & Description\\\hline
0 &  & False &  &  & Status\\\hline
 & 0 & & & & Armed\\\hline
 & 1 & & & & \\\hline
\end{tabular}
\subsection{ID 60 - Arm Recovery}
Frequency: 0Hz\\
\begin{tabular}{ |p{0.05\linewidth}|p{0.05\linewidth}|p{0.1\linewidth}|p{0.15\linewidth}|p{0.2\linewidth}|p{0.3\linewidth}| }\hline
Byte & Bit & Signed & Range & Units & Description\\\hline
0 &  & False &  &  & Status\\\hline
 & 0 & & & & Arm Recovery\\\hline
\end{tabular}
\subsection{ID 61 - Detatch Second Stage}
Frequency: 0Hz\\
\begin{tabular}{ |p{0.05\linewidth}|p{0.05\linewidth}|p{0.1\linewidth}|p{0.15\linewidth}|p{0.2\linewidth}|p{0.3\linewidth}| }\hline
Byte & Bit & Signed & Range & Units & Description\\\hline
0 &  & False &  &  & Status\\\hline
 & 0 & & & & Detatch Second Stage\\\hline
\end{tabular}
\subsection{ID 100 - Helium Pressure PT Data}
Frequency: 50Hz\\
\begin{tabular}{ |p{0.05\linewidth}|p{0.05\linewidth}|p{0.1\linewidth}|p{0.15\linewidth}|p{0.2\linewidth}|p{0.3\linewidth}| }\hline
Byte & Bit & Signed & Range & Units & Description\\\hline
0-3 &  & False &  & Milliseconds & UTC time\\\hline
4-5 &  & False &  & PSIG & Helium Pressure\\\hline
6-7 &  & False &  & ADC counts & Raw Helium Pressure Measurement\\\hline
\end{tabular}
\subsection{ID 101 - LOX Pressure PT Data}
Frequency: 50Hz\\
\begin{tabular}{ |p{0.05\linewidth}|p{0.05\linewidth}|p{0.1\linewidth}|p{0.15\linewidth}|p{0.2\linewidth}|p{0.3\linewidth}| }\hline
Byte & Bit & Signed & Range & Units & Description\\\hline
0-3 &  & False &  & Milliseconds & UTC time\\\hline
4-5 &  & False &  & PSIG & LOX Pressure\\\hline
6-7 &  & False &  & ADC counts & Raw LOX Pressure Measurement\\\hline
\end{tabular}
\subsection{ID 102 - Ethanol Pressure PT Data}
Frequency: 50Hz\\
\begin{tabular}{ |p{0.05\linewidth}|p{0.05\linewidth}|p{0.1\linewidth}|p{0.15\linewidth}|p{0.2\linewidth}|p{0.3\linewidth}| }\hline
Byte & Bit & Signed & Range & Units & Description\\\hline
0-3 &  & False &  & Milliseconds & UTC time\\\hline
4-5 &  & False &  & PSIG & Ethanol Pressure\\\hline
6-7 &  & False &  & ADC counts & Raw Ethanol Pressure Measurement\\\hline
\end{tabular}
\subsection{ID 103 - Chamber Pressure PT Data}
Frequency: 50Hz\\
\begin{tabular}{ |p{0.05\linewidth}|p{0.05\linewidth}|p{0.1\linewidth}|p{0.15\linewidth}|p{0.2\linewidth}|p{0.3\linewidth}| }\hline
Byte & Bit & Signed & Range & Units & Description\\\hline
0-3 &  & False &  & Milliseconds & UTC time\\\hline
4-5 &  & False &  & PSIG & Chamber Pressure\\\hline
6-7 &  & False &  & ADC counts & Raw Chamber Pressure Measurement\\\hline
\end{tabular}
\subsection{ID 200 - Helium Fill Valve Hall Effect State}
Frequency: 10Hz\\
\begin{tabular}{ |p{0.05\linewidth}|p{0.05\linewidth}|p{0.1\linewidth}|p{0.15\linewidth}|p{0.2\linewidth}|p{0.3\linewidth}| }\hline
Byte & Bit & Signed & Range & Units & Description\\\hline
0-3 &  & False &  & Milliseconds & UTC time\\\hline
4 &  & False &  & Open/Closed & Helium Fill Valve Hall Effect State\\\hline
\end{tabular}
\subsection{ID 201 - LOX Fill Valve Hall Effect State}
Frequency: 10Hz\\
\begin{tabular}{ |p{0.05\linewidth}|p{0.05\linewidth}|p{0.1\linewidth}|p{0.15\linewidth}|p{0.2\linewidth}|p{0.3\linewidth}| }\hline
Byte & Bit & Signed & Range & Units & Description\\\hline
0-3 &  & False &  & Milliseconds & UTC time\\\hline
4 &  & False &  & Open/Closed & LOX Fill Valve Hall Effect State\\\hline
\end{tabular}
\subsection{ID 202 - Ethanol Fill Valve Hall Effect State}
Frequency: 10Hz\\
\begin{tabular}{ |p{0.05\linewidth}|p{0.05\linewidth}|p{0.1\linewidth}|p{0.15\linewidth}|p{0.2\linewidth}|p{0.3\linewidth}| }\hline
Byte & Bit & Signed & Range & Units & Description\\\hline
0-3 &  & False &  & Milliseconds & UTC time\\\hline
4 &  & False &  & Open/Closed & Ethanol Fill Valve Hall Effect State\\\hline
\end{tabular}
\subsection{ID 250 - LOX Tank Liquid Level Data}
Frequency: 10Hz\\
\begin{tabular}{ |p{0.05\linewidth}|p{0.05\linewidth}|p{0.1\linewidth}|p{0.15\linewidth}|p{0.2\linewidth}|p{0.3\linewidth}| }\hline
Byte & Bit & Signed & Range & Units & Description\\\hline
0-3 &  & False &  & Milliseconds & UTC time\\\hline
4 &  & False &  & Percent & LOX Tank Liquid Level\\\hline
5-6 &  & False &  & Femtofarads & Raw LOX Tank Liquid Level Measurement\\\hline
\end{tabular}
\subsection{ID 251 - Ethanol Tank Liquid Level Data}
Frequency: 10Hz\\
\begin{tabular}{ |p{0.05\linewidth}|p{0.05\linewidth}|p{0.1\linewidth}|p{0.15\linewidth}|p{0.2\linewidth}|p{0.3\linewidth}| }\hline
Byte & Bit & Signed & Range & Units & Description\\\hline
0-3 &  & False &  & Milliseconds & UTC time\\\hline
4 &  & False &  & Percent & Ethanol Tank Liquid Level\\\hline
5-6 &  & False &  & Femtofarads & Raw Ethanol Tank Liquid Level Measurement\\\hline
\end{tabular}
\subsection{ID 300 - LOX Tank Temperature Data}
Frequency: 10Hz\\
\begin{tabular}{ |p{0.05\linewidth}|p{0.05\linewidth}|p{0.1\linewidth}|p{0.15\linewidth}|p{0.2\linewidth}|p{0.3\linewidth}| }\hline
Byte & Bit & Signed & Range & Units & Description\\\hline
0-3 &  & False &  & Milliseconds & UTC time\\\hline
4-5 &  & True &  & Celcius & LOX Tank Temperature\\\hline
6-7 &  & False &  & ADC counts & Raw LOX Tank Temperature Measurement\\\hline
\end{tabular}
\subsection{ID 301 - Ethanol Tank Temperature Data}
Frequency: 10Hz\\
\begin{tabular}{ |p{0.05\linewidth}|p{0.05\linewidth}|p{0.1\linewidth}|p{0.15\linewidth}|p{0.2\linewidth}|p{0.3\linewidth}| }\hline
Byte & Bit & Signed & Range & Units & Description\\\hline
0-3 &  & False &  & Milliseconds & UTC time\\\hline
4-5 &  & True &  & Celcius & Ethanol Tank Temperature\\\hline
6-7 &  & False &  & ADC counts & Raw Ethanol Tank Temperature Measurement\\\hline
\end{tabular}
\subsection{ID 302 - Nozzle Temperature Data}
Frequency: 10Hz\\
\begin{tabular}{ |p{0.05\linewidth}|p{0.05\linewidth}|p{0.1\linewidth}|p{0.15\linewidth}|p{0.2\linewidth}|p{0.3\linewidth}| }\hline
Byte & Bit & Signed & Range & Units & Description\\\hline
0-3 &  & False &  & Milliseconds & UTC time\\\hline
4-5 &  & True &  & Celcius & Nozzle Temperature\\\hline
6-7 &  & False &  & ADC counts & Raw Nozzle Temperature Measurement\\\hline
\end{tabular}
\subsection{ID 303 - Upper Air Frame Temperature Data}
Frequency: 10Hz\\
\begin{tabular}{ |p{0.05\linewidth}|p{0.05\linewidth}|p{0.1\linewidth}|p{0.15\linewidth}|p{0.2\linewidth}|p{0.3\linewidth}| }\hline
Byte & Bit & Signed & Range & Units & Description\\\hline
0-3 &  & False &  & Milliseconds & UTC time\\\hline
4-5 &  & True &  & Celcius & Upper Air Frame Temperature\\\hline
6-7 &  & False &  & ADC counts & Raw Upper Air Frame Temperature Measurement\\\hline
\end{tabular}
\subsection{ID 304 - ITC Temperature Data}
Frequency: 10Hz\\
\begin{tabular}{ |p{0.05\linewidth}|p{0.05\linewidth}|p{0.1\linewidth}|p{0.15\linewidth}|p{0.2\linewidth}|p{0.3\linewidth}| }\hline
Byte & Bit & Signed & Range & Units & Description\\\hline
0-3 &  & False &  & Milliseconds & UTC time\\\hline
4-5 &  & True &  & Celcius & ITC Temperature\\\hline
6-7 &  & False &  & ADC counts & Raw ITC Temperature Measurement\\\hline
\end{tabular}
\subsection{ID 305 - Lower Air Frame Temperature Data}
Frequency: 10Hz\\
\begin{tabular}{ |p{0.05\linewidth}|p{0.05\linewidth}|p{0.1\linewidth}|p{0.15\linewidth}|p{0.2\linewidth}|p{0.3\linewidth}| }\hline
Byte & Bit & Signed & Range & Units & Description\\\hline
0-3 &  & False &  & Milliseconds & UTC time\\\hline
4-5 &  & True &  & Celcius & Lower Air Frame Temperature\\\hline
6-7 &  & False &  & ADC counts & Raw Lower Air Frame Temperature Measurement\\\hline
\end{tabular}
\subsection{ID 400 - Helium Pressure PT Current}
Frequency: 10Hz\\
\begin{tabular}{ |p{0.05\linewidth}|p{0.05\linewidth}|p{0.1\linewidth}|p{0.15\linewidth}|p{0.2\linewidth}|p{0.3\linewidth}| }\hline
Byte & Bit & Signed & Range & Units & Description\\\hline
0-3 &  & False &  & Milliseconds & UTC time\\\hline
4-5 &  & True &  & milliamps & Helium Pressure PT Current\\\hline
6-7 &  & False &  & ADC counts & Raw Helium Pressure PT Current Measurement\\\hline
\end{tabular}
\subsection{ID 401 - LOX Pressure PT Current}
Frequency: 10Hz\\
\begin{tabular}{ |p{0.05\linewidth}|p{0.05\linewidth}|p{0.1\linewidth}|p{0.15\linewidth}|p{0.2\linewidth}|p{0.3\linewidth}| }\hline
Byte & Bit & Signed & Range & Units & Description\\\hline
0-3 &  & False &  & Milliseconds & UTC time\\\hline
4-5 &  & True &  & milliamps & LOX Pressure PT Current\\\hline
6-7 &  & False &  & ADC counts & Raw LOX Pressure PT Current Measurement\\\hline
\end{tabular}
\subsection{ID 402 - Ethanol Pressure PT Current}
Frequency: 10Hz\\
\begin{tabular}{ |p{0.05\linewidth}|p{0.05\linewidth}|p{0.1\linewidth}|p{0.15\linewidth}|p{0.2\linewidth}|p{0.3\linewidth}| }\hline
Byte & Bit & Signed & Range & Units & Description\\\hline
0-3 &  & False &  & Milliseconds & UTC time\\\hline
4-5 &  & True &  & milliamps & Ethanol Pressure PT Current\\\hline
6-7 &  & False &  & ADC counts & Raw Ethanol Pressure PT Current Measurement\\\hline
\end{tabular}
\subsection{ID 403 - Chamber Pressure PT Current}
Frequency: 10Hz\\
\begin{tabular}{ |p{0.05\linewidth}|p{0.05\linewidth}|p{0.1\linewidth}|p{0.15\linewidth}|p{0.2\linewidth}|p{0.3\linewidth}| }\hline
Byte & Bit & Signed & Range & Units & Description\\\hline
0-3 &  & False &  & Milliseconds & UTC time\\\hline
4-5 &  & True &  & milliamps & Chamber Pressure PT Current\\\hline
6-7 &  & False &  & ADC counts & Raw Chamber Pressure PT Current Measurement\\\hline
\end{tabular}
\subsection{ID 404 - Helium Fill Valve Hall Effect Current}
Frequency: 10Hz\\
\begin{tabular}{ |p{0.05\linewidth}|p{0.05\linewidth}|p{0.1\linewidth}|p{0.15\linewidth}|p{0.2\linewidth}|p{0.3\linewidth}| }\hline
Byte & Bit & Signed & Range & Units & Description\\\hline
0-3 &  & False &  & Milliseconds & UTC time\\\hline
4-5 &  & True &  & milliamps & Helium Fill Valve Hall Effect Current\\\hline
6-7 &  & False &  & ADC counts & Raw Helium Fill Valve Hall Effect Current Measurement\\\hline
\end{tabular}
\subsection{ID 405 - LOX Fill Valve Hall Effect Current}
Frequency: 10Hz\\
\begin{tabular}{ |p{0.05\linewidth}|p{0.05\linewidth}|p{0.1\linewidth}|p{0.15\linewidth}|p{0.2\linewidth}|p{0.3\linewidth}| }\hline
Byte & Bit & Signed & Range & Units & Description\\\hline
0-3 &  & False &  & Milliseconds & UTC time\\\hline
4-5 &  & True &  & milliamps & LOX Fill Valve Hall Effect Current\\\hline
6-7 &  & False &  & ADC counts & Raw LOX Fill Valve Hall Effect Current Measurement\\\hline
\end{tabular}
\subsection{ID 406 - Ethanol Fill Valve Hall Effect Current}
Frequency: 10Hz\\
\begin{tabular}{ |p{0.05\linewidth}|p{0.05\linewidth}|p{0.1\linewidth}|p{0.15\linewidth}|p{0.2\linewidth}|p{0.3\linewidth}| }\hline
Byte & Bit & Signed & Range & Units & Description\\\hline
0-3 &  & False &  & Milliseconds & UTC time\\\hline
4-5 &  & True &  & milliamps & Ethanol Fill Valve Hall Effect Current\\\hline
6-7 &  & False &  & ADC counts & Raw Ethanol Fill Valve Hall Effect Current Measurement\\\hline
\end{tabular}
\subsection{ID 407 - Upper Air Frame VIN Current}
Frequency: 10Hz\\
\begin{tabular}{ |p{0.05\linewidth}|p{0.05\linewidth}|p{0.1\linewidth}|p{0.15\linewidth}|p{0.2\linewidth}|p{0.3\linewidth}| }\hline
Byte & Bit & Signed & Range & Units & Description\\\hline
0-3 &  & False &  & Milliseconds & UTC time\\\hline
4-5 &  & True &  & Milliamps & Upper Air Frame Board Current\\\hline
6-7 &  & False &  & ADC counts & Raw Upper Air Frame Board Current Measurement\\\hline
\end{tabular}
\subsection{ID 408 - ITC VIN Current}
Frequency: 10Hz\\
\begin{tabular}{ |p{0.05\linewidth}|p{0.05\linewidth}|p{0.1\linewidth}|p{0.15\linewidth}|p{0.2\linewidth}|p{0.3\linewidth}| }\hline
Byte & Bit & Signed & Range & Units & Description\\\hline
0-3 &  & False &  & Milliseconds & UTC time\\\hline
4-5 &  & True &  & Milliamps & ITC Board Current\\\hline
6-7 &  & False &  & ADC counts & Raw ITC Board Current Measurement\\\hline
\end{tabular}
\subsection{ID 409 - Lower Air Frame VIN Current}
Frequency: 10Hz\\
\begin{tabular}{ |p{0.05\linewidth}|p{0.05\linewidth}|p{0.1\linewidth}|p{0.15\linewidth}|p{0.2\linewidth}|p{0.3\linewidth}| }\hline
Byte & Bit & Signed & Range & Units & Description\\\hline
0-3 &  & False &  & Milliseconds & UTC time\\\hline
4-5 &  & True &  & Milliamps & Lower Air Frame Board Current\\\hline
6-7 &  & False &  & ADC counts & Raw Lower Air Frame Board Current Measurement\\\hline
\end{tabular}
\subsection{ID 500 - Upper Air Frame VIN Voltage}
Frequency: 10Hz\\
\begin{tabular}{ |p{0.05\linewidth}|p{0.05\linewidth}|p{0.1\linewidth}|p{0.15\linewidth}|p{0.2\linewidth}|p{0.3\linewidth}| }\hline
Byte & Bit & Signed & Range & Units & Description\\\hline
0-3 &  & False &  & Milliseconds & UTC time\\\hline
4-5 &  & True &  & Millivolts & Upper Air Frame Board VIN Voltage\\\hline
6-7 &  & False &  & ADC counts & Raw Upper Air Frame Board VIN Voltage Measurement\\\hline
\end{tabular}
\subsection{ID 501 - ITC VIN Voltage}
Frequency: 10Hz\\
\begin{tabular}{ |p{0.05\linewidth}|p{0.05\linewidth}|p{0.1\linewidth}|p{0.15\linewidth}|p{0.2\linewidth}|p{0.3\linewidth}| }\hline
Byte & Bit & Signed & Range & Units & Description\\\hline
0-3 &  & False &  & Milliseconds & UTC time\\\hline
4-5 &  & True &  & Millivolts & ITC Board VIN Voltage\\\hline
6-7 &  & False &  & ADC counts & Raw ITC Board VIN Voltage Measurement\\\hline
\end{tabular}
\subsection{ID 502 - Lower Air Frame VIN Voltage}
Frequency: 10Hz\\
\begin{tabular}{ |p{0.05\linewidth}|p{0.05\linewidth}|p{0.1\linewidth}|p{0.15\linewidth}|p{0.2\linewidth}|p{0.3\linewidth}| }\hline
Byte & Bit & Signed & Range & Units & Description\\\hline
0-3 &  & False &  & Milliseconds & UTC time\\\hline
4-5 &  & True &  & Millivolts & Lower Air Frame Board VIN Voltage\\\hline
6-7 &  & False &  & ADC counts & Raw Lower Air Frame Board VIN Voltage Measurement\\\hline
\end{tabular}
% WARNING END OF SECTION AUTOGENERATED BY PYTHON SCRIPT
% THIS SECTION MAY BE AUTOMATICALLY CHANGED AT ANY TIME
% Autogenerated section name: CAN Config
% File path of script: ARD/genCAN.py


\end{document}
